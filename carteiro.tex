
    %:LLPStartPreview para rodar o pdf com mudancas automaticas

\documentclass[12pt, a4paper]{article}
\usepackage{graphicx}
\usepackage{wrapfig}
\usepackage[utf8]{inputenc}
\usepackage[brazil]{babel} % Separacao de silabas em portugues
\usepackage{amsthm} % has proof
\usepackage{amsmath}
\usepackage{listings}
\usepackage{tcolorbox}
\usepackage {tikz}
\usepackage{hyperref}
\usepackage{float}
\usetikzlibrary {positioning}
\usetikzlibrary{arrows}
%\usepackage {xcolor}

\tikzset{edge/.style = {->,> = latex'}}

    \graphicspath{
    {.} % document root dir
    {./img/}
}

\renewcommand\refname{Referências}
\newtheorem{theorem}{Teorema}[section]
\newtheorem{corollary}{Corolário}
\newtheorem{lemma}{Lema}

\title{O problema do carteiro chinês}
\author{Orientador: Carlos Eduardo Ferreira\\Gabriel Fernandes de Oliveira}
\date{}

\begin{document}

        \textbf{Heurística da arborescência geradora mínima}

        Para guiar a explicação da heurística toma-se, como exemplo, o digrafo $G$, a seguir:
   

        \begin{figure}[H]
            \centering
            \begin{tikzpicture}[node distance=3cm, every loop/.style={},thick,main node/.style={circle,draw,font=\sffamily\Large}]

                \node[main node] at (-5, 0) (f) {f};
                \node[main node] at (-2, 3) (e) {e};
                \node[main node] at (2, 3) (c) {c};
                \node[main node] at (5, 0) (a) {a};
                \node[main node] at (-2, -3) (d) {d};
                \node[main node] at (2, -3) (b) {b};

                \path[->] (a) edge[line width=2, bend right, below] node {4} (b);
                \path[->] (a) edge[left] node {4} (c);
                \path[->] (b) edge[line width=2, below, bend right] node {5} (a);
                \path[->] (c) edge[line width=2, above] node {8} (e);
                \path[->] (c) edge[above, left] node {2} (b);
                \path[->] (d) edge[below] node {6} (b);
                \path[->] (d) edge[line width=2, left] node {5} (f);
                \path[->] (e) edge[below] node {4} (f);
                \path[->] (e) edge[pos=0.4, left] node {8} (d);
                \path[->] (f) edge[above] node {1} (b);
            \end{tikzpicture}
            \caption{Digrafo $G$, o conjunto $R$ corresponde aos arcos em negrito}
            \label{pcr-digraph}
        \end{figure}

        A extensão de $G$ em um digrafo euleriano $G' = (V', A')$ se dá de modo similar àquela do caso com grafos não direcionados:

        O conjunto de vértices possui a mesma definição, $V' = \{u \in V : (u, v) \in R \text{ para algum } v \in V\}$. 

        Como no grafo da figura \ref{pcr-digraph} todos vértices possuem ao menos um arco pertencente à $R$, define-se $V' = V$.

        De modo semelhante, o conjunto de arcos $A'$ será inicialmente acrescido dos arcos $(v_i, v_j)$ para todo par $v_i, v_j \in V'$, de custo igual ao custo do menor caminho de $v_i$ a $v_j$.
        Posteriormente remove-se de $A'$ todo arco que pertence também a $A \setminus R$ e cujo custo $c_{uv}$ é igual a $c_{uw} + c_{wv}$ para algum vértice $w$, removem-se também os arcos paralelos de mesmo valor que pertencem a $A \setminus R$.

        Realizando tal extensão no exemplo sugerido, teremos o seguinte digrafo $G'$:

        \begin{figure}[H]
            \centering
            \begin{tikzpicture}[node distance=3cm, every loop/.style={},thick,main node/.style={circle,draw,font=\sffamily\Large}]

                \node[main node] at (-5, 0) (f) {f};
                \node[main node] at (-2, 3) (e) {e};
                \node[main node] at (2, 3) (c) {c};
                \node[main node] at (5, 0) (a) {a};
                \node[main node] at (-2, -3) (d) {d};
                \node[main node] at (2, -3) (b) {b};

                \path[->] (a) edge[line width=2, bend right, below] node {4} (b);
                \path[->] (a) edge[left] node {4} (c);
                \path[->] (b) edge[line width=2, below, bend right] node {5} (a);
                \path[->] (c) edge[line width=2, above] node {8} (e);
                \path[->] (c) edge[above, left, pos=0.4] node {2} (b);
                \path[->] (d) edge[below] node {6} (b);
                \path[->] (d) edge[line width=2, left] node {5} (f);
                \path[->] (e) edge[below] node {4} (f);
                \path[->] (e) edge[pos=0.4, left] node {8} (d);
                \path[->] (f) edge[above] node {1} (b);

                \path[->] (a) edge[red, below, bend left=100] node {20} (d);
                \path[->] (a) edge[red, above, pos=0.6] node {16} (f);
                \path[->] (b) edge[bend left, red, below] node {25} (d);
                \path[->] (b) edge[red, right, pos=0.7] node {17} (e);
                \path[->] (b) edge[red, above, bend right=15] node {21} (f);
                \path[->] (e) edge[red, bend left, above] node {14} (c);
                \path[->] (f) edge[red, above, bend left=100] node {10} (c);
                \path[->] (f) edge[red, below, bend right] node {26} (d);
                \path[->] (f) edge[red, above, bend left] node {18} (e);


            \end{tikzpicture}
            \caption{Digrafo $G'$, extensão de $G$, as arestas criadas na extensão são mostradas em vermelho.}
        \end{figure}
		O grafo $G'$ será composto por $k$ conjuntos conexos $G_1, G_2, \dots, G_k$, induzidos por $R$. 
		Isto é, cada conjunto $G_i$ será composto apenas por arcos pertencentes a $R$ e será conexo, mas não necessariamente fortemente conexo.

        No exemplo apresentado, $G$ possui três componentes, $G_1$ composta pelos vértices $a, b$, $G_2$ composta por $c, e$ e $G_3$ composta por $d, f$.


	\begin{enumerate}
        \item[\textbf{Passo 1.}] 
			Encontrar uma arborescência geradora $T$ de custo mínimo que conecte os subgrafos $G_1, \dots, G_k$ e tem raiz em um vértice qualquer.

            Para encontrar tal arborescência, pode-se utilizar o algoritmo proposto independentemente pela dupla Yoeng-Jin Chu e Tseng-Hong Liu e por Edmonds\cite{edmonds-ssa}, sendo, portanto, chamado de algoritmo de Chu-Liu/Edmonds.

            Condensando o digrafo $G'$ em suas components $G_1, G_2, G_3$ temos o seguinte grafo:


            \begin{figure}[H]
                \centering
                \begin{tikzpicture}[node distance=3cm, every loop/.style={},thick,main node/.style={circle,draw,font=\sffamily\Large}]

                    \node[main node, dashed] at (4, -4) (g1) {\parbox{1.5cm}{\centering $G_1$}};
                    \node[main node, dashed, line width=2] at (0, 3) (g2) {\parbox{1.5cm}{\centering $G_2$}};
                    \node[main node, dashed] at (-4, -4) (g3) {\parbox{1.5cm}{\centering $G_3$}};


                    \path[->] (g3) edge[below] node {6} (g1);
                    \path[->] (g3) edge[below, bend left=15] node {1} (g1);
                    \path[->] (g1) edge[below, bend left=15, red] node {25} (g3);
                    \path[->] (g1) edge[below, bend left, red] node {20} (g3);
                    \path[->] (g1) edge[below, bend left=45, red] node {16} (g3);
                    \path[->] (g1) edge[below, bend left=60, red] node {21} (g3);
                    \path[->] (g3) edge[left, bend left=15, red] node {18} (g2);
                    \path[->] (g3) edge[left, bend left, red] node {10} (g2);
                    \path[->] (g2) edge[left] node {4} (g3);
                    \path[->] (g2) edge[bend left=15, left] node {8} (g3);
                    \path[->] (g2) edge[right] node {2} (g1);
                    \path[->] (g1) edge[right, red, bend left=15] node {17} (g2);
                    \path[->] (g1) edge[right, bend right=15] node {4} (g2);

                \end{tikzpicture}
                \caption{Digrafo $G'$ condensado em suas componentes induzidas por $R$}
                \label{condensated-dig}
            \end{figure}

                O primeiro passo do algoritmo de Chu-Liu/Edmonds consiste em definir um vértice raiz $r$ para a arborescência.
                Como a heurística de Christofides não requer que um vértice específico seja tal raiz, podemos escolher o vértice condensado $G_1$ para cumprir tal função ($r = G_1$). 

                Definida a raiz $r$, deve-se retirar do digrafo analisado todos arcos que tem como destino $r$.
                Além disso, pode-se também substituir qualquer conjunto de arcos paralelos por um único arco com custo igual ao menor custo dos arcos paralelos removidos. 
                Pode-se visualizar o efeito de tais modificações em $G'$ na figura \ref{chu-liu}.

            \begin{figure}[H]
                \centering
                \begin{tikzpicture}[node distance=3cm, every loop/.style={},thick,main node/.style={circle,draw,font=\sffamily\Large}]

                    \node[main node, dashed] at (4, -4) (g1) {\parbox{2cm}{\centering $r = G_1$}};
                    \node[main node, dashed, line width=2] at (0, 3) (g2) {\parbox{1.5cm}{\centering $G_2$}};
                    \node[main node, dashed] at (-4, -4) (g3) {\parbox{1.5cm}{\centering $G_3$}};


                    \path[->] (g1) edge[below, red] node {16} (g3);
                    \path[->] (g3) edge[left, bend left=10, red] node {10} (g2);
                    \path[->] (g2) edge[left, bend left=10] node {4} (g3);
                    \path[->] (g1) edge[right] node {4} (g2);

                \end{tikzpicture}
                \caption{Digrafo $G'$ após a remoção de arcos sugeridas pelo algoritmo de Chu-Liu/Edmonds}
                \label{chu-liu}
            \end{figure}

            Para todo vértice $v$ do grafo condensado diferente da raiz ($v \neq r$) encontra-se o arco de menor custo que chega em $v$.
            Define-se como $\pi(v)$ o vértice origem de tal arco.

            Se o conjunto de arcos $T = \{(\pi(v), v) | v \neq r\}$ não contem circuitos, então $T$ é uma arborescência de custo mínimo enraizada em $r$.
            Do contrário, realiza-se uma contração dos circuitos existentes em $T$, atualizam-se os custos dos arcos e repete-se recursivamente o mesmo procedimento de criação de $T$.

            No exemplo analisado, o conjunto $T$ consiste nos arcos de $G_1$ a $G_2$ e $G_2$ a $G_3$, ambos de custo 4.

            \begin{figure}[H]
                \centering
                \begin{tikzpicture}[node distance=3cm, every loop/.style={},thick,main node/.style={circle,draw,font=\sffamily\Large}]

                    \node[main node, dashed] at (4, -4) (g1) {\parbox{2cm}{\centering $r = G_1$}};
                    \node[main node, dashed, line width=2] at (0, 3) (g2) {\parbox{1.5cm}{\centering $G_2$}};
                    \node[main node, dashed] at (-4, -4) (g3) {\parbox{1.5cm}{\centering $G_3$}};


                    \path[->] (g1) edge[below, red] node {16} (g3);
                    \path[->] (g3) edge[left, bend left=10, red] node {10} (g2);
                    \path[->] (g2) edge[blue, left, bend left=10] node {4} (g3);
                    \path[->] (g1) edge[blue, right] node {4} (g2);

                \end{tikzpicture}
                \caption{Conjunto de arcos $T$ definidos pelo algoritmo de Chu-Liu/Edmonds em azul.}
                \label{chu-liu-p}
            \end{figure}


            Como $T$ não possui circuitos, não é necessário realizar uma nova iteração do algoritmo.
            $T$ é o conjunto de arcos que induz a arborescência de custo mínimo enraizada em $G_1$.

            Define-se como $R \cup T$ o digrafo induzido pelos arcos de $R$ e da arborescência $T$.


        \begin{figure}[H]
            \centering
            \begin{tikzpicture}[node distance=3cm, every loop/.style={},thick,main node/.style={circle,draw,font=\sffamily\Large}]

                \node[main node] at (-5, 0) (f) {f};
                \node[main node] at (-2, 3) (e) {e};
                \node[main node] at (2, 3) (c) {c};
                \node[main node] at (5, 0) (a) {a};
                \node[main node] at (-2, -3) (d) {d};
                \node[main node] at (2, -3) (b) {b};

                \path[->] (a) edge[line width=2, bend right, below] node {4} (b);
                \path[->] (a) edge[blue, left] node {4} (c);
                \path[->] (b) edge[line width=2, below, bend right] node {5} (a);
                \path[->] (c) edge[line width=2, above] node {8} (e);
                \path[->] (d) edge[line width=2, left] node {5} (f);
                \path[->] (e) edge[blue, below] node {4} (f);
            \end{tikzpicture}
            \caption{Digrafo $R\cup T$, os arcos de $T$ são representados em azul}
            \label{graphRUT}
        \end{figure}



        \item[\textbf{Passo 2.}]
            Encontrar um multiconjunto $M$ de menor custo composto por arcos de $A'$ que torna o digrafo induzido por $R \cup T$ euleriano, ou seja iguala os graus de entrada e saída de todos vértices.

			Pode-se determinar $M$ a partir da resolução de um problema de transporte: 
            O problema de transporte será definido tendo como base o grafo $G'$, porém as funções de oferta e demanda são definidas a partir dos graus dos vértices no grafo induzido por $R \cup T$.

            Um vértice $v \in V(G')$ cujo grau de entrada é maior que seu grau de saída (em relação ao subgrafo $R \cup T$) possuirá uma oferta igual ao valor absoluto da diferença de seus graus, do contrário, o valor absoluto da diferença representará a demanda do vértice $v$.

            No exemplo abordado, figura \ref{graphRUT}, o vértice $f$ possui uma oferta de valor $2$, os vértices $a$ e $d$ possuem uma demanda de valor $1$ e os vértices restantes já se encontram em igualdade de grau de entrada e saída.

			A resolução do problema de transporte modelado gera um conjunto de caminhos, representando a distribuição ótima da oferta. 
            O multiconjunto dos arcos pertencentes a união dos caminhos que solucionam o problema é o multiconjunto $M$ desejado.

            Sendo assim, como apenas uma origem existe no problema apresentado como exemplo, a solução do mesmo consiste na utilização dos caminhos de menor custo de $f$ a $a$ e de $f$ a $d$. 

        \begin{figure}[H]
            \centering
            \begin{tikzpicture}[node distance=3cm, every loop/.style={},thick,main node/.style={circle,draw,font=\sffamily\Large}]

                \node[main node] at (-5, 0) (f) {f};
                \node[main node] at (-2, 3) (e) {e};
                \node[main node] at (2, 3) (c) {c};
                \node[main node] at (5, 0) (a) {a};
                \node[main node] at (-2, -3) (d) {d};
                \node[main node] at (2, -3) (b) {b};

                \path[->] (a) edge[line width=2, bend right, below] node {4} (b);
                \path[->] (a) edge[left] node {4} (c);
                \path[->] (b) edge[line width=2, below, bend right] node {5} (a);
                \path[->] (c) edge[line width=2, above] node {8} (e);
                \path[->] (d) edge[line width=2, left] node {5} (f);
                \path[->] (e) edge[below] node {4} (f);

                \path[->] (f) edge[red, below] node {1} (b);
                \path[->] (b) edge[red, below, bend right=60] node {5} (a);
                \path[->] (f) edge[red, bend right, below] node {26} (d);
            \end{tikzpicture}
            \caption{Digrafo eurleriano $R\cup T \cup M$ com arcos de $M$ representados em vermelho.}
            \label{graphRUTUM}
        \end{figure}

        Na figura \ref{graphRUTUM} pode-se visualizar o digrafo induzido pelos arcos $R\cup T \cup M$. 
        Note que o arco de $b$ a $a$ de custo 5 é representado duas vezes, isso pois ele pertence tanto a $R$ quanto a $M$.

        \item[\textbf{Passo 3.}]
			O digrafo induzido pelo multiconjunto de arcos $R \cup T \cup M$ é, pela definição de $M$, euleriano. 
			
			Portanto, pode-se derivar um circuito euleriano de tal grafo. Por sua vez, a partir deste circuito, é possível derivar uma solução do PCR para o grafo original $G$.

            Um possível circuito euleriano para o exemplo apresentado é representado na figura \ref{circ-eulerianoRUTUM} a seguir.


        \begin{figure}[H]
            \centering
            \begin{tikzpicture}[node distance=3cm, every loop/.style={},thick,main node/.style={circle,draw,font=\sffamily\Large}]

                \node[main node] at (-5, 0) (f) {f};
                \node[main node] at (-2, 3) (e) {e};
                \node[main node] at (2, 3) (c) {c};
                \node[main node] at (5, 0) (a) {a};
                \node at (6, 0) (ini) {início};
                \node[main node] at (-2, -3) (d) {d};
                \node[main node] at (2, -3) (b) {b};

                \path[->] (a) edge[line width=2, bend right, below] node {1} (b);
                \path[->] (b) edge[line width=2, below, bend right] node {2} (a);
                \path[->] (a) edge[left] node {3} (c);
                \path[->] (c) edge[line width=2, above] node {4} (e);
                \path[->] (e) edge[below] node {5} (f);
                \path[->] (f) edge[bend right, below] node {6} (d);
                \path[->] (d) edge[line width=2, left] node {7} (f);
                \path[->] (f) edge[below] node {8} (b);
                \path[->] (b) edge[below, bend right=60] node {9} (a);
            \end{tikzpicture}
            \caption{Circuito euleriano de $R\cup T \cup M$.}
            \label{circ-eulerianoRUTUM}
        \end{figure}

        A partir de tal circuito pode-se encontrar uma solução para o PCR do digrafo original $G$ expandindo os arcos contraídos na construção de $G'$ e removendo as duplicatas de arcos criados no procedimento da heurística.
        Por exemplo, o arco do nó $f$ ao $d$ de custo $26$ pertencente a $M$ consiste na condensação do caminho mínimo de $f$ a $d$: $\{f, b, a, c, e, d\}$.

        Realizando a remoção dos arcos artificiais adicionados a $G$ chegamos no seguinte circuito, baseado no circuito euleriano mostrado na figura \ref{circ-eulerianoRUTUM}:

        \begin{figure}[H]
            \centering
            \begin{tikzpicture}[node distance=3cm, every loop/.style={},thick,main node/.style={circle,draw,font=\sffamily\Large}]

                \node[main node] at (-5, 0) (f) {f};
                \node[main node] at (-2, 3) (e) {e};
                \node[main node] at (2, 3) (c) {c};
                \node[main node] at (5, 0) (a) {a};
                \node at (6, 0) (ini) {início};
                \node[main node] at (-2, -3) (d) {d};
                \node[main node] at (2, -3) (b) {b};

                \path[->] (a) edge[line width=2, bend right, below] node {1} (b);
                \path[->] (b) edge[line width=2, below, bend right, sloped] node {2, 7, 13} (a);
                \path[->] (a) edge[left] node {3, 8} (c);
                \path[->] (c) edge[line width=2, above] node {4, 9} (e);
                \path[->] (e) edge[below] node {5} (f);

                \path[->] (e) edge[right] node {10} (d);

                \path[->] (d) edge[line width=2, left] node {11} (f);
                \path[->] (f) edge[pos=0.6, below, sloped] node {6, 12} (b);
                \path[->] (c) edge[dashed, above, left] node {} (b);
                \path[->] (d) edge[dashed, below] node {} (b);

            \end{tikzpicture}
            \caption{Construção de um circuito que resolve o PCR do grafo $G$. Representam-se em pontilhado os arcos de $G$ não percorridos na solução desenvolvida.}
            \label{solucaoPCR}
        \end{figure}
	\end{enumerate}

    Finaliza-se assim a execução da heurística da arborescência geradora mínima, que gera uma solução para o PCR de $G$ com custo 62.

	\begin{thebibliography}{9}
	\bibitem{konigsberg} 
	Euler, Leonhard
	\textit{Solution problematis ad geometriam situs pertinentis}. 
	Comment. Acad. Sci. U. Petrop 8, 128–40, 1736.

	\bibitem{hierholzer}
	Hierholzer, Carl
	\textit{``Über die Möglichkeit, einen Linienzug ohne Wiederholung und ohne Unterbrechung zu umfahren''}, 
	Mathematische Annalen, 6 (1): 30–32, doi:10.1007/BF01442866, 1873.

    \bibitem{tanya}
    Problema D do round \#288 (Div. 2) retirado do Codeforces\\
    \href{https://codeforces.com/contest/508/problem/D}{codeforces.com/contest/508/problem/D}

    \bibitem{tanya-sol}
    Solução para o problema Tanya and Password, desenvolvida em C++\\
    \href{https://github.com/gafeol/competitive-programming/blob/master/ojs/cf/508/D.cpp}{github.com/gafeol/competitive-programming/blob/master/ojs/cf/508/D\\.cpp}

    \bibitem{sereja}
    Problema C do round \#215 (Div. 1) retirado do Codeforces\\
    \href{https://codeforces.com/problemset/problem/367/C}{codeforces.com/problemset/problem/367/C}

    \bibitem{sereja-sol}
    Solução para o problema Sereja and the Arrangement of Numbers, desenvolvida em C++\\
    \href{https://github.com/gafeol/competitive-programming/blob/master/ojs/cf/367/C.cpp}{github.com/gafeol/competitive-programming/blob/master/ojs/cf/367/C.cpp}

    \bibitem{jogging}
    Problema 10296 retirado do UVa\\
    \href{https://onlinejudge.org/index.php?option=com_onlinejudge&Itemid=8&page=show_problem&problem=1237}{onlinejudge.org/index.php?option=com\_onlinejudge\&Itemid=8\&page=show\\\_problem\&problem=1237}

    \bibitem{jogging-sol}
    Solução para o problema Jogging Trails, desenvolvida em C++\\
    \href{https://github.com/gafeol/competitive-programming/blob/master/ojs/UVa/1237.cpp}{github.com/gafeol/competitive-programming/blob/master/ojs/UVa/1237.cpp} 

    \bibitem{mit}
    Exemplo retirado do site do MIT, exercício 6.6.c\\
    \href{http://web.mit.edu/urban_or_book/www/book/chapter6/problems6/6.6.html}{web.mit.edu/urban\_or\_book/www/book/chapter6/problems6/6.6.html} 

    \bibitem{wiki-christofides}
    Página da wikipedia sobre o algoritmo de Christofides\\
    \href{https://en.wikipedia.org/wiki/Christofides_algorithm}{wikipedia.org/wiki/Christofides\_algorithm}

    \bibitem{michel}
    Artigo ``Arc Routing Problems Part II: The Rural Postman Problem'' publicado por Michel Gendreau e Gilbert Laporte.
    \href{https://pubsonline.informs.org/doi/10.1287/opre.43.3.399}{pubsonline.informs.org/doi/10.1287/opre.43.3.399}


    \bibitem{christofides-86}
    Christofides, Nicos, et al. "An algorithm for the rural postman problem on a directed graph." Netflow at pisa. Springer, Berlin, Heidelberg, 1986. 155-166.\\
    \href{https://link.springer.com/chapter/10.1007/BFb0121091}{link.springer.com/chapter/10.1007/BFb0121091}


    \bibitem{edmonds-ssa}
    \href{https://nvlpubs.nist.gov/nistpubs/jres/71B/jresv71Bn4p233_A1b.pdf<Paste>}{nvlpubs.nist.gov/nistpubs/jres/71B/jresv71Bn4p233\_A1b.pdf}    
	\end{thebibliography}
 
\end{document}
