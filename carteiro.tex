
    %:LLPStartPreview para rodar o pdf com mudancas automaticas

\documentclass[12pt, a4paper]{article}
\usepackage{graphicx}
\usepackage{wrapfig}
\usepackage[utf8]{inputenc}
\usepackage[brazil]{babel} % Separacao de silabas em portugues
\usepackage{amsthm} % has proof
\usepackage{amsmath}
\usepackage{listings}
\usepackage{tcolorbox}
\usepackage {tikz}
\usepackage{hyperref}
\usepackage{float}
\usetikzlibrary {positioning}
\usetikzlibrary{arrows}
%\usepackage {xcolor}

\tikzset{edge/.style = {->,> = latex'}}

    \graphicspath{
    {.} % document root dir
    {./img/}
}

\renewcommand\refname{Referências}
\newtheorem{theorem}{Teorema}[section]
\newtheorem{corollary}{Corolário}
\newtheorem{lemma}{Lema}

\title{O problema do carteiro chinês}
\author{Orientador: Carlos Eduardo Ferreira\\Gabriel Fernandes de Oliveira}
\date{}

\begin{document}



        \section{PCR em grafos direcionados}

        Para tratar sobre a versão direcionada do PCR, tomaremos $G = (V, A)$ como o digrafo original e um conjunto $R \subseteq A$ de arcos que devem ser percorridos no circuito que resolve o PCR para $G$.

        Assim como no PCR para grafos não direcionados, a solução deste caso se baseia em estender o digrafo $G$ em um digrafo euleriano, cujo circuito euleriano representa uma solução para o PCR do digrafo original $G$.

        A estensão de $G$ em $G' = (V', A')$ se dá de modo similar àquela do caso anterior:

        O conjunto de vértices possui a mesma definição, $V' = \{u \in V : (u, v) \in R \text{ para algum } v \in V\}$. 
        De mesmo modo, $A'$ será inicialmente acrescida dos arcos $(v_i, v_j)$ para todo par $v_i, v_j \in V'$, de custo igual ao custo do menor caminho de $v_i$ a $v_j$.
        Posteriormente remove-se de $A'$ todo arco que pertence também a $A \setminus R$ e cujo custo $c_{uv}$ é igual a $c_{uw} + c_{wv}$ para algum vértice $w$, removem-se também os arcos paralelos de mesmo valor que pertencem a $A \setminus R$.

		O grafo $G'$ será composto por $k$ conjuntos conexos $G_1, G_2, \dots, G_k$, induzidos por $R$. 
		Isto é, cada conjunto $G_i$ será composto apenas por arcos pertencentes a $R$ e será conexo, mas não necessariamente fortemente conexos.

        Será apresentada agora uma heurística que resolve esse problema, proposta por Christofides \cite{christofides-86}, muito semelhente à heurística apresentada para o caso de grafos não direcionados.

	\textbf{Heurística da arborescência geradora mínima}

	\begin{enumerate}
        \item[\textbf{Passo 1.}] 
			Encontrar uma arborescência geradora $T$ de custo mínimo que conecte os subgrafos $G_1, \dots, G_k$ e tem raiz em um vértice qualquer.

            Para encontrar tal arborescência, pode-se utilizar o algoritmo proposto independentemente pela dupla Yoeng-Jin Chu e Tseng-Hong Liu e por Edmonds\cite{edmonds-ssa}, sendo, portanto, chamado de algoritmo de Chu-Liu/Edmonds.

			Seja $R \cup T$ o subgrafo de $G'$ induzido pelos arcos de $R$ e da arborescência $T$.

        \item[\textbf{Passo 2.}]
            Encontrar um multiconjunto $M$ de menor custo composto por arcos de $A'$ que torna o subgrafo induzido por $R \cup T$ euleriano, ou seja iguala os graus de entrada e saída de todos vértices.

			Pode-se determinar $M$ a partir da resolução de um problema de transporte, segue a modelagem de tal problema: 
            O problema de transporte será definido tendo como base o grafo $G'$, porém as funções de oferta e demanda são definidas a partir dos graus dos vértices no grafo induzido por $R \cup T$.
            Um vértice $v \in V(G')$ cujo grau de entrada é maior que seu grau de saída (em relação ao subgrafo $R \cup T$) possuirá uma oferta igual ao valor absoluto da diferença dos graus, do contrário, o valor absoluto da diferença representará a demanda do vértice $v$.

			A resolução deste problema gera um conjunto de caminhos, representando a distribuição ótima da oferta. O multiconjunto dos arcos pertencentes aos caminhos que solucionam o problema é o multiconjunto $M$ desejado.

        \item[\textbf{Passo 3.}]
			O digrafo induzido pelo multiconjunto de arcos $R \cup T \cup M$ é, pela definição de $M$, euleriano. 
			
			Portanto, pode-se derivar um circuito euleriano de tal grafo, a partir deste circuito euleriano, por sua vez, é possível derivar uma solução do PCR para o grafo original $G$.
	\end{enumerate}

	\begin{thebibliography}{9}
	\bibitem{konigsberg} 
	Euler, Leonhard
	\textit{Solution problematis ad geometriam situs pertinentis}. 
	Comment. Acad. Sci. U. Petrop 8, 128–40, 1736.

	\bibitem{hierholzer}
	Hierholzer, Carl
	\textit{``Über die Möglichkeit, einen Linienzug ohne Wiederholung und ohne Unterbrechung zu umfahren''}, 
	Mathematische Annalen, 6 (1): 30–32, doi:10.1007/BF01442866, 1873.

    \bibitem{tanya}
    Problema D do round \#288 (Div. 2) retirado do Codeforces\\
    \href{https://codeforces.com/contest/508/problem/D}{codeforces.com/contest/508/problem/D}

    \bibitem{tanya-sol}
    Solução para o problema Tanya and Password, desenvolvida em C++\\
    \href{https://github.com/gafeol/competitive-programming/blob/master/ojs/cf/508/D.cpp}{github.com/gafeol/competitive-programming/blob/master/ojs/cf/508/D\\.cpp}

    \bibitem{sereja}
    Problema C do round \#215 (Div. 1) retirado do Codeforces\\
    \href{https://codeforces.com/problemset/problem/367/C}{codeforces.com/problemset/problem/367/C}

    \bibitem{sereja-sol}
    Solução para o problema Sereja and the Arrangement of Numbers, desenvolvida em C++\\
    \href{https://github.com/gafeol/competitive-programming/blob/master/ojs/cf/367/C.cpp}{github.com/gafeol/competitive-programming/blob/master/ojs/cf/367/C.cpp}

    \bibitem{jogging}
    Problema 10296 retirado do UVa\\
    \href{https://onlinejudge.org/index.php?option=com_onlinejudge&Itemid=8&page=show_problem&problem=1237}{onlinejudge.org/index.php?option=com\_onlinejudge\&Itemid=8\&page=show\\\_problem\&problem=1237}

    \bibitem{jogging-sol}
    Solução para o problema Jogging Trails, desenvolvida em C++\\
    \href{https://github.com/gafeol/competitive-programming/blob/master/ojs/UVa/1237.cpp}{github.com/gafeol/competitive-programming/blob/master/ojs/UVa/1237.cpp} 

    \bibitem{mit}
    Exemplo retirado do site do MIT, exercício 6.6.c\\
    \href{http://web.mit.edu/urban_or_book/www/book/chapter6/problems6/6.6.html}{web.mit.edu/urban\_or\_book/www/book/chapter6/problems6/6.6.html} 

    \bibitem{wiki-christofides}
    Página da wikipedia sobre o algoritmo de Christofides\\
    \href{https://en.wikipedia.org/wiki/Christofides_algorithm}{wikipedia.org/wiki/Christofides\_algorithm}

    \bibitem{michel}
    Artigo ``Arc Routing Problems Part II: The Rural Postman Problem'' publicado por Michel Gendreau e Gilbert Laporte.
    \href{https://pubsonline.informs.org/doi/10.1287/opre.43.3.399}{pubsonline.informs.org/doi/10.1287/opre.43.3.399}


    \bibitem{christofides-86}
    Christofides, Nicos, et al. "An algorithm for the rural postman problem on a directed graph." Netflow at pisa. Springer, Berlin, Heidelberg, 1986. 155-166.\\
    \href{https://link.springer.com/chapter/10.1007/BFb0121091}{link.springer.com/chapter/10.1007/BFb0121091}


    \bibitem{edmonds-ssa}
    \href{https://nvlpubs.nist.gov/nistpubs/jres/71B/jresv71Bn4p233_A1b.pdf<Paste>}{nvlpubs.nist.gov/nistpubs/jres/71B/jresv71Bn4p233\_A1b.pdf}    
	\end{thebibliography}
 
\end{document}
