\documentclass[]{article}
\usepackage[brazil]{babel}
\usepackage[utf8]{inputenc}

\title{Relatório de IC\\Problema do Carteiro Chinês}
\date{}

\begin{document}
\maketitle

Durante esse semestre foi desenvolvido um relatório que conta atualmente com 30 páginas estudando o problema do caminho euleriano, do carteiro chinês e aplicações da teoria apresentada em problemas de competições de programação.

Inicialmente o relatório trata do problema dos caminhos eulerianos, assunto importante para o estudo do problema do carteiro chinês. Nesta seção inicial é apresentada a motivação histórica do problema, os principais teoremas do assunto e uma solução algoritmica para o problema.

Segue uma seção sobre o problema do carteiro chinês que é subdividida em três seções, cada uma tratando de uma variação do problema: a aplicação do problema em grafos não direcionados, em grafos direcionados e em grafos mistos. 

A última seção trata das aplicações da teoria apresentada no trabalho em problemas usados em competições universitárias de programação. 
Cada subseção trata de um diferente problema, contando com a explicação da solução do mesmo e ainda uma referência para um código que soluciona o problema. 

Pretende-se continuar a pesquisa expandindo o estudo para outras variações do problema do carteiro chinês, adicionando ao relatório a variação rural e a hierárquica, estudando novas aplicações da teoria apresentada em problemas e implementando as soluções do problema do caminho euleriano e das variações de grafos não direcionados e direcionados no problema do carteiro chinês.


\end{document}
