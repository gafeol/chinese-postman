%:LLPStartPreview para rodar o pdf com mudancas automaticas
% vim-tex cheatsheet (https://github.com/lervag/vimtex):
    % \ll - start compiling document
    % \lv - open pdf viewer
    % \ll or \lk stop running compilation
    % \le parse log files for errors and warnings
% \lc remove auxiliary files

\documentclass[12pt, a4paper]{article}
\usepackage[brazil]{babel} % Separacao de silabas em portugues
\usepackage[noend]{algpseudocode}
\usepackage[utf8]{inputenc}
\usepackage{algorithm}
\usepackage{amsfonts}
\usepackage{amsmath}
\usepackage{amsthm} % has proof
\usepackage{array}
\usepackage{float}
\usepackage{graphicx}
\usepackage{hyperref}
\usepackage{import}
\usepackage{listings}
\usepackage{tcolorbox}
\usepackage{tikz}
\usepackage{wrapfig}
\usetikzlibrary {positioning}
\usetikzlibrary{arrows}
%\usepackage {xcolor}

\algrenewcommand\algorithmicend{\textbf{fim}}
\algrenewcommand\algorithmicdo{\textbf{faça}}
\algrenewcommand\algorithmicwhile{\textbf{enquanto}}
\algrenewcommand\algorithmicfor{\textbf{para}}
\algrenewcommand\algorithmicif{\textbf{se}}
\algrenewcommand\algorithmicthen{\textbf{então}}
\algrenewcommand\algorithmicelse{\textbf{senão}}
\algrenewcommand\algorithmicreturn{\textbf{devolve}}
\algrenewcommand\algorithmicfunction{\textbf{função}}
\algrenewcommand\algorithmicprocedure{\textbf{procedimento}}
\algrenewcommand\algorithmicrepeat{\textbf{repita}}
\algrenewcommand\algorithmicuntil{\textbf{enquanto}}




\tikzset{edge/.style = {->,> = latex'}}

    \graphicspath{
    {.} % document root dir
    {./img/}
}

\renewcommand\refname{Referências}
\newtheorem{theorem}{Teorema}[section]
\newtheorem{corollary}{Corolário}
\newtheorem{lemma}{Lema}

% Definicoes de graus 
\DeclareMathOperator{\gr}{\delta}
\DeclareMathOperator{\gre}{\delta^+}
\DeclareMathOperator{\grs}{\delta^-}
\DeclareMathOperator{\grt}{\delta_t}

%% BIBTEX
\usepackage{csquotes}
\usepackage[
backend=biber,
style=alphabetic,
sorting=nty,
maxbibnames=10
]{biblatex}
\addbibresource{ref.bib}
\renewbibmacro*{volume+number+eid}{%
  \printfield{volume}%
  \setunit*{\addnbspace}% NEW (optional); there's also \addnbthinspace
  \printfield{number}%
  \setunit{\addcomma\space}%
  \printfield{eid}}
\DeclareFieldFormat[article]{number}{\mkbibparens{#1}}
%%%

\begin{document}

% ---------------------------------------------------------------------------- %
% CAPA
% Nota: O título para as dissertações/teses do IME-USP devem caber em um 
% orifício de 10,7cm de largura x 6,0cm de altura que há na capa fornecida pela SPG.
\thispagestyle{empty}
\begin{center}
    \vspace*{2.3cm}
    \textbf{\huge{O problema do Carteiro Chinês}}\\
    
    \vspace*{1cm}
    \Large{Gabriel Fernandes de Oliveira}

    \vskip 1.8cm
    Orientador: Carlos Eduardo Ferreira\\

    \vspace{\fill}
    \normalsize{São Paulo, 2020}
\end{center}

% ---------------------------------------------------------------------------- %

\newpage
\tableofcontents
\newpage

\import{/}{euler.tex}

\import{/}{pcc.tex}

\import{/}{prb.tex}

    \iffalse
        \section{Anotações}

        \begin{itemize}
            \item Todo mixed CPP pode ser transformado em um WPP.
        \end{itemize}

    \fi

	\medskip

    \newpage
    \printbibliography
 
\end{document}
