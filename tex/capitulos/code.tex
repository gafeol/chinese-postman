\chapter{Código desenvolvido}

Estão disponíveis no GitHub as implementações para todos os algoritmos apresentados na monografia.

Para acessá-los, basta acessar a pasta \textit{code} do \href{github.com/gafeol/chinese-postman/}{repositório}.

Nesse capítulo explica-se a organização do código e discute-se mais a fundo as implementações realizadas.

\section{Estruturas de dados}

São disponibilizados códigos gerais para estruturas de dados, focados na área de grafos:
\begin{itemize}
    \item \texttt{aresta.hpp} 
    \item \texttt{aresta-ingreme.hpp}
    \item \texttt{grafo.hpp} 
    \item \texttt{digrafo.hpp}
    \item \texttt{grafo-misto.hpp}
    \item \texttt{grafo-ingreme.hpp}
\end{itemize}

Além disso disponibilizam-se também estruturas de dados gerais como:

\begin{itemize}
    \item \texttt{union-find.cpp}
\end{itemize}

\section{Algoritmos auxiliares}

\begin{itemize}
    \item \texttt{floyd-warshall.cpp}
    \item \texttt{problema-transporte.cpp}
    \item \texttt{min-cost-flow.cpp}
\end{itemize}

\begin{itemize}
    \item \texttt{ssa.cpp}
    \item \texttt{min-cost-matching/MCM.cpp}
\end{itemize}

\section{Implementações}

\subsection{Euler}

\begin{itemize}
    \item \texttt{euler-grafo.cpp}
    \item \texttt{euler-digrafo.cpp}
    \item \texttt{euler-misto.cpp}
\end{itemize}

\subsection{Problema do Carteiro Chinês}
\begin{itemize}
    \item \texttt{pcc-grafo.cpp}
    \item \texttt{pcc-digrafo.cpp}
    \item \texttt{pcc-misto.cpp}
    \item \texttt{pcr-grafo.cpp}
    \item \texttt{pcr-digrafo.cpp}
    \item \texttt{pcv-guan.cpp}
\end{itemize}

\section{Testes}
