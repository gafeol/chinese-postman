\chapter{Código desenvolvido}

Estão disponíveis no GitHub as implementações para todos os algoritmos apresentados na monografia.

Para acessá-los, basta acessar a pasta \textit{code} do \href{github.com/gafeol/chinese-postman/}{repositório}.

Nesse capítulo explica-se a organização do código e discute-se mais a fundo as implementações realizadas.

\section{Estruturas de dados}

São disponibilizados códigos gerais para estruturas de dados, focados na área de grafos:
\begin{itemize}
    \item \texttt{aresta.hpp} 

        Define a \texttt{struct Aresta}, usada para modelar tanto aresta direcionadas quanto não-direcionadas.

    Toda aresta possui um valor  \texttt{prox} indicando o vértice com o qual ela se liga, um identificador \texttt{id} e um custo real \texttt{cus}.
    \item \texttt{aresta-ingreme.hpp}

        Define a \texttt{struct ArestaIngreme}, modelando uma aresta que possui custos diferentes dependendo do sentido em que é percorrida.

        Além de armazenar os mesmos valores \texttt{prox} e \texttt{id} de uma aresta comum, também armazena \texttt{dirCus} e \texttt{invCus}, custos para se percorrer uma aresta na direção do vértice \texttt{prox} e no sentido inverso, a partir de \texttt{prox}, respectivamente.

    \item \texttt{grafo.hpp} 
        
        Define a \texttt{struct Grafo}, com \texttt{n} vértices e \texttt{m} arestas, armazenadas em uma lista de adjacências \texttt{adj}.

        Utiliza a \texttt{struct Aresta} para representar as arestas do grafo.

    \item \texttt{digrafo.hpp}

        Analogamente, define a \texttt{struct Digrafo}, com \texttt{n} vértices e \texttt{m} arcos, armazenados em uma lista de adjacências \texttt{adj}.

        Também utiliza a \texttt{struct Aresta} para representar os arcos do digrafo.

    \item \texttt{grafo-misto.hpp}

        Define a \texttt{struct Misto} que representa grafos que possuem tanto arestas quanto arcos. 

        Além das propriedades \texttt{n, m, adj}, esta estrutura contem um contador \texttt{nArestas} que conta a quantidade de arestas no grafo.

        Isso é especialmente útil para determinar se uma aresta com identificador \texttt{id} qualquer é uma aresta (\texttt{id $<$ nArestas}) ou um arco (\texttt{id $\geq$ nArestas}).

        Para facilitar a manipulação desta estrutura, a \texttt{struct Misto} contêm métodos auxiliares como:
        \begin{itemize}
            \item  Devolver o grau total (\texttt{grauTotal(v)}), grau de entrada (\texttt{grauEntrada(v)}) e saída (\texttt{grauSaida(v)}) de todo vértice
            \item Contar o número de componentes fortemente conexas, usando o algoritmo de Tarjan (\texttt{countSCC()})
            \item Checar se uma aresta, dada um identificador é arco (\texttt{arco(id)}) ou aresta (\texttt{aresta(id)})
        \end{itemize}

    \item \texttt{grafo-ingreme.hpp}

    Define a \texttt{struct GrafoIngreme} usada na modelagem do problema do carteiro com vento.

    Este grafo utiliza a estrutura \texttt{ArestaIngreme} para representar suas arestas.

\end{itemize}

Além disso disponibilizam-se também estruturas de dados gerais como:

\begin{itemize}
    \item \texttt{union-find.cpp}

        Essa implementação do Union Find utiliza as otimizações de compressão de caminhos, encurtando as relações de ancestralidade sempre que possível, e a união das componentes de menor tamanho sob as componentes maiores.
        Atingindo assim uma complexidade $O(\log^*)$ amortizada por operações de busca e união.
\end{itemize}

\section{Algoritmos auxiliares}

\begin{itemize}
    \item \texttt{floyd-warshall.cpp}
    \item \texttt{problema-transporte.cpp}
    \item \texttt{min-cost-flow.cpp}
\end{itemize}

\begin{itemize}
    \item \texttt{ssa.cpp}
    \item \texttt{min-cost-matching/MCM.cpp}
\end{itemize}

\section{Implementações}

\subsection{Euler}

\begin{itemize}
    \item \texttt{euler-grafo.cpp}
    \item \texttt{euler-digrafo.cpp}
    \item \texttt{euler-misto.cpp}
\end{itemize}

\subsection{Problema do Carteiro Chinês}
\begin{itemize}
    \item \texttt{pcc-grafo.cpp}
    \item \texttt{pcc-digrafo.cpp}
    \item \texttt{pcc-misto.cpp}
    \item \texttt{pcr-grafo.cpp}
    \item \texttt{pcr-digrafo.cpp}
    \item \texttt{pcv-guan.cpp}
\end{itemize}

\section{Testes}
