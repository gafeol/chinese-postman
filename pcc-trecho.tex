\section{O problema do Carteiro Chinês (PCC)}


\subsection{Grafos não direcionados}
        \label{sec:pcc}
        [...]

    %TODO: Adicionar valor da solucao otima
    \subsection{Problema do carteiro chinês com vento}

        Quando houver em um grafo uma aresta que possui custos diferentes dependendo do sentido em que a mesma é percorrida estamos tratando do Problema do Carteiro Chinês com Vento (PCCV).

        A motivação dessa variação é que podem existir estradas com uma corrente de vento que o carteiro deve percorrer. 
        A corrente de vento tanto pode ajudar o carteiro se ele anda no mesmo sentido que a corrente, quanto pode atrapalhar o carteiro se ele anda no sentido contrário do vento.

        Outra ilustração possível para esta variação é imaginar que uma estrada pode ser inclinada, sendo assim mais difícil subir tal estada do que descé-la.

        Esta variação do problema do carteiro chinês também é NP-difícil.

        Porém, como provado por Meigu Guan \cite{guan-windy}, se todo circuito de um grafo $G$ possui o mesmo custo independente do sentido em que é percorrido, então é possível encontrar uma solução para o PCCV usando um algoritmo polinomial, trataremos esse caso a seguir.

        Seja $c_{ij}$ o custo de se percorrer a aresta $ij$ de $i$ a $j$ e $c_{ji}$ o custo de percorrer a mesma aresta no sentido contrário.

        O algoritmo sugerido por Guan se baseia em criar uma nova função de custo $c'$ para o problema, tal que $c'_{ij} = c'_{ji} = \frac{c_{ij} + c_{ji}}{2}$ para toda aresta $ij$. 
        Essa transformação reduz o problema a um PCC não direcionado, que possui solução polinomial, como tratado na seção \ref{sec:pcc}.

        O teorema \ref{windy-theorem} enunciado a seguir, garante que uma resposta ótima do PCC no grafo $(G, c')$ equivale a uma resposta ótima, de mesmo custo, do PCCV no grafo original $(G, c)$.

        \begin{lemma}
            \label{lemma:pccv}
            Seja $C$ um circuito de $G$, vale que:

            \[
                \sum_{ij \in C} c'_{ij} = \sum_{ij \in C} c'_{ji} =  \sum_{ij \in C} c_{ij} =  \sum_{ij \in C} c_{ji}
            \]
        \end{lemma}
        \begin{proof}
            Como $G$ respeita a propriedade que todo circuito possui custo igual, não importando a ordem em que o mesmo é percorrido, pode-se afirmar que:
            \[
                \sum_{ij \in C} c_{ij} =  \sum_{ij \in C} c_{ji}
            \]

            Além disso, por definição $c'_{ij} = c'_{ji}$, valendo assim que:
            \[
                \sum_{ij \in C} c'_{ij} = \sum_{ij \in C} c'_{ji} 
            \]

            Finalmente, pela definição de $c'$: 
            \begin{align*}
                \sum_{ij \in C} c'_{ij} &= \sum_{ij \in C} \frac{c_{ij} + c_{ji}}{2} \\
                \sum_{ij \in C} c'_{ij} &= \frac{1}{2} \left(\sum_{ij \in C}c_{ij} + \sum_{ij \in C}c_{ji} \right) \\
                \sum_{ij \in C} c'_{ij} &= \sum_{ij \in C} c_{ij}
            \end{align*}

        \end{proof}


        \begin{theorem}
            \label{windy-theorem}
            Seja $G$ um grafo conexo não direcionado cujos circuitos possuem o mesmo custo, independente do sentido que são percorridos. 

            Qualquer solução ótima para o PCC em $(G, c')$ também é uma solução ótima do PCCV em $(G, c)$.

        \end{theorem}
        \begin{proof} 
            Seja $T$ uma solução para o PCC em $(G, c')$. 
            $T$ será um passeio fechado composto por arestas orientadas de $G$.

            Pode-se dividir $T$ em circuitos de arcos $C_1, C_2, \dots, C_k$, com $k \in \mathbb{N}$ . % TODO: provar que pode dividir T em circuitos
            Podemos expressar o custo da solução $T$ como a soma dos custos de cada arco dos circuitos $C_1, C_2, \dots, C_k$. 

            Provaremos por indução, que para um circuito qualquer $C_i$ vale que:
            \[
                \sum_{ij \in C_i} c_{ij} = \sum_{ij \in C_i} c'_{ij}
            \]

            A indução é realizada com base no tamanho do circuito $C_i$ analisado.

            Há dois casos base desta indução:

            \begin{itemize}
                \item Quando o circuito analisado $C_i$ é vazio.
                \item Quando $C_i$ é um circuito que é presente no grafo original $G$. 
                    Isto é, $C_i$ não percorre uma mesma aresta de $G$ mais que uma vez.
                    O lema \ref{lemma:pccv} garante a hipótese de igualdade neste caso.
            \end{itemize}

            Assume-se que a hipótese de indução vale para circuitos com até $k$ arcos.

            Seja $C_i$ um circuito que não faz parte do caso base e que possui $k+1$ arcos.

            Pela definição de circuitos, $C_i$ não pode possuir arcos duplicados.  % TODO: talvez esteja um pouco confusa essa questao de ser um circuito porem nao ser circuito de $G$
            Sendo assim, para $C_i$ não estar presente em $G$, deverá existir uma aresta $uv$ de $G$ tal que $uv \in C_i$ e $vu \in C_i$, como representa-se a seguir:

            \[
                C_i = \{ w_1w_2, \dots, w_iu, uv, vw_{i+1}, \dots, w_jv,  vu, uw_{j+1}, \dots w_kw_1\}
            \]

            É possível, a partir de tal representação retirar de $C_i$ os arcos $uv$ e $vu$, separando-o em dois circuitos direcionados de tamanho menor:

            \[
                C'_i = \{w_1w_2, \dots w_iu, uw_{j+1}, \dots, w_kw_1\} 
            \]
            \[
                C''_i = \{vw_{i+1}, \dots, w_jv\}
            \]

            Como ambos novos circuitos tem tamanho menor que $|C_i| = k+1$, vale a hipótese para ambos:
            \begin{align}
                \sum_{ij \in C'_i} c_{ij} &= \sum_{ij \in C'_i} c'_{ij} \\
                \sum_{ij \in C''_i} c_{ij} &= \sum_{ij \in C''_i} c'_{ij}
            \end{align}
            
            Além disso, vale pela definição da função $c'$, que:
            
            \begin{align}
                c_{uv} + c_{vu}  &= c'_{uv} + c'_{vu} 
            \end{align}

            Somando as três igualdades apresentadas temos a representação do custo de $C_i$:

            \begin{align*}
                \sum_{ij \in C'_i} c_{ij} +  \sum_{ij \in C''_i} c_{ij} + c_{uv} + c_{vu}  &= \sum_{ij \in C'_i} c'_{ij} + \sum_{ij \in C''_i} + c'_{ij} c'_{uv} + c'_{vu}  \\
                \sum_{ij \in C_i} c_{ij} &= \sum_{ij \in C_i} c'_{ij} \\
            \end{align*}
             
            Provando assim que a hipótese de indução também vale para um circuito de tamanho $k+1$.

            Finalmente, temos que o custo da solução $T$ do PCCV de $(G, c)$ será igual ao custo da solução do PCC em $(G, c')$:

            \begin{align*}
                \sum_{ij \in T} c_{ij} &= \sum_{1 \leq u \leq k} \sum_{ij \in C_u} c_{ij} \\
                                       &= \sum_{1 \leq u \leq k} \sum_{ij \in C_u} c'_{ij} \\
                                       &= \sum_{ij \in T} c'_{ij} \\
            \end{align*}

        \end{proof}

%    \subsection{Problema do Carteiro Hierárquico}
%
%    Nesta variação são definidos $k$ subconjuntos de arestas $\{A_1, A_2, \dots, A_k\}$.
%    Como no problema do cartiro chinês original deve-se encontrar um passeio que passe por todas arestas de um grafo $G$, porém esse passeio só poderá percorrer uma aresta pertencente a um subconjunto $A_j$ se o mesmo já tiver percorrido todas arestas pertencentes aos conjuntos $A_i$ com $1 \leq i < j$.
%
%    Esta variação também é NP-difícil.
